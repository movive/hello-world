\subsection{试验废水来源及水质分析}
项目水源来自\showprojectlocation的焦化废水进水。城北污水处理厂位于滨江大道和经三路交叉口西南片,远期占地约210亩,根据《城北污水处理厂初步设计说明书》其远期污水处理服务范围是针对循环经济试验园72家企业进行工业废水处理。根据项目批复和建设计划近期(2010年)处理规模8万m³/d,远期(2020年)日处理规模16万m³/d。园区内有7家企业的工业废水多年以来一直由企业自行处理达到一定的标准经由二冶大沟排入长江,
目前政府正对其排污管网进行改造,预计在2017年11月30日完成,改造完成以后,其工业废水即将纳入城北污水处理厂,其中包括铜陵泰富排放的焦化废水。项目前期取焦化废水做生化处理后,进出水水质如表\ref{tab1}所示。\par
\begin{center}
\captionof{table}{水处理工艺进出水水质情况\quad单位(mg/L)}
\label{tab1}
\begin{tabular}{| c | c | c | c | c | c | c | c | c | c |}
    \hline
    序号 & 水样来源 & COD & NH3-N & NO2-N & NO3-N & TP & SS & pH & BOD\\ \hline 
    1 & 进水 & 244 & 4.31 & 0.033 & 0.03 & 0.423 & 298 & 7.67 & 68.1\\ \hline 
    2 & 出水 & 187 & 7.23 & 0.027 & 0.15 & 0.325 & 26.4 & 7.58 & 10.7\\ \hline 
\end{tabular}
\end{center}
\setlength{\parindent}{2em}
\par
可以看出,除COD外,其他各项指标均满足污水综合排放标准。由于出水\linebreak COD在180mg/L左右,达不到一级指标中要求的低于50mg/L。为达标排放,亟需进行深度处理以满足污水排放要求。需要对其单独进行高级氧化处理后,方能保证出水达标。\par

\subsection{试验方案}
本次试验废水设计进水水质COD$_{Cr}$=200mg/L,BOD$_5$=40mg/L,BOD/COD=0.20属于不可生化的污水。其中,总氮,总磷,氨氮等指标进水即符合排放标准,因此重点去除项目主要为SS和COD。\par
(1)去SS\par
要求出水SS浓度小于20mg/L,去除率为90\%,要求的去除率很高。根据国内外现有资料,煤层气采出水去除悬浮物的方法大多采用物理方法或物化方法去除。混凝、沉淀、过滤是含悬浮物采出水净化过程中必不可少的单元过程,混凝、沉淀操作不仅可去除原水中95\%以上的悬浮物和90\%左右的细菌,还能保证未能沉淀的细小矾花在后续的过滤操作中能被滤料截留。这种处理方式,不仅流程简单,处理效果也较好。\par
(2)除COD\par
要求出水COD浓度小于50mg/L,去除率为66.7\%,要求的去除率非常高。由于废水的生化性较差,无法采用常规生化方法处理,要达到COD66.7\%的去除效果,需要进行高级氧化处理打断有机大分子间的C-C键,提高可生化效率,做到出水达标。\par
综上分析可知,既有混凝效果又能氧化有机污染物的芬顿工艺满足水质处理需求是煤化工废水合理有效的处理手段。本试验的主体工艺采用高效Fenton氧化+混凝沉淀。工艺段包括酸碱调节段、Fenton氧化段、中和及混凝沉淀段。试验步骤如下:\par
(1)将废水pH调节至Fenton反应所需的最佳pH条件,保证Fenton反应的处理效果。\par
(2)采用Fenton系统对废水进行深度氧化处理,该技术的主要原理是外加的H$_2$O$_2$氧化剂与Fe$^{2+}$催化剂,即所谓的Fenton药剂, 两者在适当的pH下会反应产生氢氧自由基(OH·),而氢氧自由基的高氧化能力与废水中的有机物反应,可分解氧化有机物,进而降低废水中难以被混凝沉淀的COD。\par
(3)Fenton出水中含有大量的亚铁及铁离子,且废水为酸性,经加入液碱后调节废水pH至中性使废水pH达到排放标准,同时投加混凝剂及助凝剂PAC、PAM去除废水较高的悬浮物。\par

