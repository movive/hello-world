本课题目标如下:\par
1.	严格执行环境保护的各项规定,确保经处理后的臭气达到排放标准;\par
2.	参照国内外现有污水处理厂废气处理工程的实际经验,选用成熟、可靠、先进的处理工艺,使先进性和可靠性有机的结合起来;\par
3.	在废气达标的前提下,尽量降低工程投资和运行费用;\par
4.	废气处理工程应操作运行与维护管理简单方便,运行稳定,占地小;\par
5.	平面布置力求在便于施工、便于安装和便于维修的前提下,使各构筑物尽量集中,节约用地,扩大绿化面积,并尽量减少对周边环境的影响;\par
6.	在自控方面,采用自动和手动相结合,遵循“集中管理、分散控制”的原则,设置必要的自控仪表和预警系统,尽量提高系统自动化和管理水平,减少人员编制;\par
7.	尽可能采用有效的方式,减小加盖施工时对污水处理厂正常运行的影响;尽量减小新增加的盖结构负荷,减小对原构筑物结构的影响;尽量采用低加盖(罩)结构形式,以减小除臭空间,减少除臭风量,节约能源,降低工程运行费用。加盖结构轻便、抗老化、耐腐蚀,同时清洁、冲洗方便。\par
