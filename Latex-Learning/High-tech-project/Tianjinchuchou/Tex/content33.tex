(1)填料选择\par
生物滤池的最主要部分是填料。一种好的填料必须满足:容许生长的微生物种类多;供微生物生长的表面积大;营养成分合理(N、P、K和痕量元素);吸水性好;自身的气味少;吸附性好;结构均匀孔隙率大;价格便宜;腐烂慢(运行时间厂、养护周期长)。单成分滤料一般只满足上述的部分要求,配方合适的多成分混合物可以较全面的满足要求。常用的填料有:干树皮、干草、纤维性泥炭或其混合物。由于填料本身是有机养分,当滤池暂停运行时,微生物可以利用填料的有机成分继续维持生命活动。本工程选择粗制活性炭和轻质陶粒组合填料。\par
(2)工艺条件控制\par
整个处理工艺包括收集和处理。为了避免气味源气味扩散,扩散源要求封闭,并使它处于负压状态。从气味源收集到的气体被送到生物滤池处理,进滤池的空气要求潮湿,相对湿度必须为80\%-95\%,否则填料会干化,微生物将失活。为了防止滤池被堵塞,必须在空气进入以前除去其中的小颗粒,所以空气进入以前要调节喷水量,维持洗涤器中气体达到所要求的湿度,用于喷淋的水是厂区回用水或者滤池本身的渗水。\par
