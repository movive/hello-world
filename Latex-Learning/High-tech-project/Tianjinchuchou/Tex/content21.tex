\par
目前,在国内外应用的除臭工艺有很多。主要有物理法、化学法和生物法。\par
物理法:恶臭物理处理方法并不改变恶臭物质的化学性质,而是用一种物质将恶臭化合物的臭味掩蔽、稀释或者将恶臭物质中气相转移至液相或者固相。常见的方法有中和法、稀释法、掩蔽法和吸附法等。\par
化学法:化学处理是通过氧化恶臭化合物,改变恶臭物质的化学结构,使之转变为无臭物质或者臭味强度较低的物质。常见的方法有氧化法、光催化降解法,燃烧法,臭氧氧化法等。\par
生物法:生物法脱臭是近几十年发展起来的一种新的废气治理技术,其最大特点是运行成本低,无二次污染。它是利用经过驯化后的微生物将恶臭物质分解为二氧化碳和水,或其它易回收物,而达到脱臭的目的。微生物在氧化分解恶臭物质的过程中,还可同时将恶臭物转变为自身的营养物,微生物得以产生细胞,继续繁殖。\par
微生物处理臭气的基本原理是利用微生物把溶解于水中的恶臭物质吸收于微生物自身体内,通过微生物的代谢活动使其降解的一种过程。\par
其他恶臭治理方法\par
(1)膜分离法:膜分离法是利用膜对废气和空气的选择透过性使废气净化。根据膜构成的不同,分为固膜和液膜分离两种。液膜分离技术可净化H2S、CO2等气体;固膜分离技术可用来回收氨,浓缩甲烷气,从C5和C5以下烷烃中分离乙烯、丙烯等。该法节能,效率高。已成功应用于化工、医药、环境保护等领域内。\par
(2)等离子体分解法:日本的植松性行利用等离子体的化学作用分解氯氟烃等难分解气体。这种技术能在较短时间内完成,并且在小型装置内进行大量废气的处理。由两个系统组成:一个利用高频等离子体急速加热等离子体,使其温度升到10000℃,然后与水蒸气接触进行分解;另一个系统用来把排气冷却到80℃。\par
(3)电晕法:在高能电子作用下产生氧化自由基O、OH;有机物分子受到高能电子碰撞被激发及原子键断裂形成小碎片基团;O、OH与激发原子有机物分子破碎的分子基团、自由基等发生反应,最终降解为CO、CO2、H2O。1988年以来,美国就开展了电晕法降解低浓度的挥发性有机物的研究。研究表明在环境通常温度和压力下,该法能达到较好的效率。\par
由于恶臭的特殊性,往往要求达到很高的脱除率,才不至于仍有使人不愉快的臭味,这时常常需要采用两种以上的净化手段,例如洗涤-吸附,燃烧-吸收,化学氧化-吸附、水吸收-化学氧化、甚至更多种作用同时进行等等,才能达到要求。\par
\par
