对建成后的高级催化氧化单元进行进水验证,持续跟踪试验8天。得到的检测结果及操作条件如下:\par
%\vspace{50mm}
\begin{center}
\captionof{table}{生产性试验检测数据}
\label{tab6}
\begin{tabular}{| c | c | c | c | c | c |}
    \hline
    日 & 硫酸亚铁(g/L) & 双氧水(g/L) & 反应时间min & 原水COD(mg/L) & 出水COD(mg/L)\\ \hline 
    1 & 1.635 & 0.8 & 120 & 81.5 & 263\\ \hline 
    2 & 1.09 & 0.8 & 120 & 90.2 & 291\\ \hline 
    3 & 1.635 & 2.67 & 120 & 54.5 & 73.5\\ \hline 
    4 & 1.09 & 2.67 & 120 & 36.5 & 68\\ \hline 
    5 & 0.65 & 1.067 & 120 & 37.1 & 91.5\\ \hline 
    6 & 0.43 & 1.067 & 120 & 60.6 & 109\\ \hline 
    7 & 0.818 & 2.67 & 120 & 35.7 & 87.5\\ \hline 
    8 & 0.654 & 2.67 & 120 & 54.6 & 77.8\\ \hline     
\end{tabular}
\end{center}
\setlength{\parindent}{2em}
\par
现场得到的结果和前期试验结果类似,焦化废水COD经过高级催化氧化单元可以处理至100mg/L以下,与其他工业水混合生化处理后出水满足当地达标排放标准。因工艺方案与运行参数及设备细节方面的改进,运行成本的吨水处理费用为12元,比按Fenton设计要求理论计算出的22元要降低50\%以上。这主要因为实施中针对废水进水SS较高的情况,优化了混凝剂的最佳投加量,改变搅拌桨的外形与安装角度设计,强化了反应液的湍动与混合效果,确保出水悬浮物含量较低;另一方面,进一步优化了Fenton氧化的工艺参数,调整了氧化剂的投加比例、投加剂量及反应前后溶液的pH,同时从机械设计角度对反应设备进行了优化,减少设备死角,增强催化剂、氧化剂与水中有机污染物的接触与反应,使催化剂、氧化剂的使用效率得到较大提高,从而大幅减少了化学药剂的投加量(处理一吨水,双氧水投加量由13L降低至6L,硫酸亚铁投加量由5.45kg降低至2.45kg),整体运行费用降至12元。\par

