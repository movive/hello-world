经过生产性实验,总结有以下几点:\par
1)铜陵福泰焦化废水水质水量波动大:焦化废水水质随日时变化有很大区别,导致废水的COD有部分偏高现象,而另有部分时间段,COD基本达到出水标准,仅SS一项不达标。此外,不同时间段SS含量区别很大,经现场统计,水质可分为黑色浑浊水、白色浑浊水,无色透明水质,尤其以黑色水质的影响最大,该类废水对工艺段的运行主要影响在需随时跟进调节混凝段加药量。\par
2)反应pH值对Fenton反应的影响较大:经对比不同pH条件下,Fenton工艺处理白色透明水质,酸度越低,处理效果越差,因此后续实验中采取不加酸碱的条件下投加Fenton试剂,降低了运行费用及操作难度和危险性。运行费用中的药剂费也由设计之初因设备设计为常规Fenton工艺,因此当来水水质波动较大时,采用正常的Fenton处理工艺可达到一定的处理效果。\par
3)浮泥问题源于水质中TDS极高,不同于常规的市政废水或工业废水,由于TDS过高导致废水密度大,因而混凝沉淀处理效果差,需降低沉淀池表面负荷,延长沉淀池的停留时间,后续解决办法只能是减小进水量;\par
4)在调试过程中Fenton处理后COD不降反升,出现该问题的主要原因未发现氯离子浓度过高的问题,影响了检测结果的准确性,问题经发现后,采用高氯法检测原水COD,检测值已经低于出水标准,因此,仅需调整Fenton工艺方式,除去废水中高浓度的硫化物,防止废水长期储存导致的发臭问题。改变Fenton工艺的运行条件后,不加酸碱的条件下,直接投加硫酸亚铁,可将废水中的硫化物转化为硫化亚铁,经双氧水进一步氧化后变为沉淀絮体,经沉淀与水分离,去除了水中的硫化物及少部分的还原性物质。\par
5)工艺设计方面存在的问题为:混凝池采用二级絮凝,絮凝效果低于三级絮凝;沉淀池的设计以普通水质考虑,未考虑高盐废水密度大,絮体难以沉降,易造成浮泥的问题;Fenton加药系统考虑设计两级加药,减少钢结构件的支出;Fenton中和池后接提升泵,也直接影响了后续沉淀性较差的问题。需在以后的设计中进行改进。\par
6)Fenton加药池设为两级加药,一级为硫酸和硫酸亚铁,二级为双氧水;中和混凝池改为三级中和,一级为液碱,二级液碱,三级为PAM;去掉提升改为自流,提升泵应改为管道泵加在Fenton反应器出水管道上。\par
