1、课题驯化水厂原生菌群,培养出除臭专型菌群,在实验中表现出良好的处理效果。\par
2、从实验中总结出进气负荷与停留时间的关系,除臭设备压降与进气速率的关系,进而可以根据进气条件推算出除臭设备体积,占地面积,风管规格,构建出除臭设备设计的基本框架。\par
3、凭借运行项目中得到的经验,我们梳理出几条设计规则作为日后参考:\par
(1)设备运行过程中若长时间停止喷淋会导致设备压降发生明显变化。设计时需要尽可能增加隔板上孔径大小,以降低运行时设备压降。\par
(2)喷淋头尽量选用塑料、陶瓷等不易结垢生锈的材料,形状选择不容易堵塞的螺旋形状。\par
(3)除臭设备耗水量大,失水的原因主要有:缝隙漏水、随上升气流吹出的液滴和加湿入口空气三个。设计时应在设备顶部加装集液网来进行收集进而滴回设备填料层,并并需要注意出气口的安装位置,避免出气口外形成冰层,造成安全隐患。\par
(4)设备放空管管径要尽量大,设于设备底部侧面,与设备底部保持一定距离,以免底部杂质被水流带入管路,引起堵塞。\par
