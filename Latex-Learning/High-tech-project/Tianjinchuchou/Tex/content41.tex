本项目为企业自主独立研究项目,用以考察\showproject中的菌剂筛选、填料的使用条件和设备运行参数。生物过滤是使收集到的废气再适宜的条件下通过长满微生物的固体载体(填料),气味物质先被填料吸收,然后被填料上的微生物氧化分解,完成废气的除臭过程。固体载体上生长的微生物承担了物质转换的任务,因为微生物生长需要足够的有机养分,所以固体载体必须具有高的有机成分。要使微生物保持高的活性,还必须为之创造一个良好的生存条件,比如:适宜的十渡、pH值、氧气含量、温度和营养成分等。环境天健变化会影响微生物的生长繁殖,因此在试运行或改变工况时要考虑生物过滤池会有一个适应过程。本课题研究重点放在填料的使用条件和设备运行参数。项目流程依据公司标准试验流程设计。项目分以下步骤进行,分别是技术调研、试验构建、最佳工艺控制参数确定、工程技术设计说明编制、生产性试验设计、试验验证。在管理上,项目严格按照公司指定的《科研课题管理办法》执行,将依次进行立项、定期小结、中期考核、验收评审的工作。