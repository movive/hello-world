\par
\qquad研究成果将有助于采用经济实用的技术处理垃圾场中的恶臭气体H2S、NH3和臭气浓度等达到国家规定的恶臭污染排放厂界一级标准,缓解垃圾场被投斥的压力。同时改善垃圾场工作人员的工作环境和附近居民的生活质量。\par
生物滴滤法处理臭气工艺主要是在生物吸收法基础上进行改进,集合了生物过滤法和生物吸收法两种工艺的优点。滴滤塔所用的填料应具有易于挂膜、不易堵塞、比表面积大等特点。在生物滴滤塔中存在一个连续流动的水相,整个传质过程涉及气、液、固3相,通过回流水可以控制滴滤池水相的pH值,为微生物提供营养元素。生物滴滤塔的反应条件(pH值、温度等)易于控制(通过调节循环液的pH值、温度),而生物滤池中pH值的控制则主要通过在装填料时加入适当的固体缓冲剂来完成。一旦缓冲剂耗尽,则需更新或再生滤料,而温度的调节则需要外加强制措施来完成。因此,在处理含硫、含氮等经微生物降解会产生酸性代谢产物及释放能量较大的污染物时,生物滴滤塔比生物滤池更有效。\par
目前,除臭工艺在市场上有很多,但都达不到垃圾场彻底除臭的需求。若本课题研究成功,我公司可应用到公司自己的填埋场和堆肥场,解决臭味问题。并可根据此产品大量抢占国内市场,迅速获取高额利润。\par
