\par近年来,我国为了优化工业资源,建设了一批经济技术开发区、特色工业园区及技术示范区等多种形式的工业园区。2016年我国建成和在建的各类工业园区数量达到了9000多个,工业废水排放量占全国污水排放总量的45\%左右。工业园区污水主要来自园区工厂生产过程中产生的污水和废液,其主要特点是:成分复杂,污染物浓度高;具有一定毒性;可生化性较差;水质不稳定。基于成本考虑,我国对工业园区的污水大多仍采用“预处理+二级生化处理”的工艺,但随国家对出水水质标准的提高,单纯的生物处理工艺已很难满足出水水质要求,因此对工业废水二级生化出水进行深度处理,进一步降低有毒有害物质的含量具有重大的现实意义。近年来各种高级氧化技术,如紫外光催化氧化技术、臭氧催化氧化技术、电催化氧化技术、超声降解技术、芬顿催化氧化技术等在有毒难降解有机废水处理方面有大量的研究工作报道。\par