芬顿氧化技术是通过Fe$^{2+}$与H$_2$O$_2$之间发生反应产生·OH来降解水中的有机污染物。·OH是强氧化性物质,能无选择地与废水中的污染物反应,降解常规生物方法无法处理的有毒有害物质,其在化工\cite{hujie_2015}、印染\cite{wangping_2015}、木薯制酒精\cite{shiqiang_2015}、含酚\cite{liyang_2017}、制药\cite{zengping_2017}等废水处理中均有应用,并取得了较好的处理效果。芬顿氧化技术处理工业废水具有反应启动快,反应条件温和、反应设备简单、能耗消耗小,经济性好等优点。虽然芬顿氧化技术具有上述优点,但其存在药剂消耗量大、污泥产量多的问题,这限制了其在工程上的大规模应用。\par
本课题拟采用强化芬顿氧化技术深度处理工业园区二级生化出水,重点解决生化处理过程无法降解的有机物的处理问题,在保证出水水质合格的前提下,对芬顿氧化技术的工艺参数进行优化,以降低污泥产量与运行成本,为工业园区废水深度处理问题提供可行的解决方案,从而为芬顿氧化技术的工程化应用提供途径。\par
