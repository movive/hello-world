\projectname{强化芬顿技术}
\manager{冯卫强}
\department{技术部}
\applydate{2017.10.16}
\company{桑德生态科技有限公司}
\managergender{男}
\managerbirth{1980.10}
\managerethnic{汉}
\managerdegree{硕士研究生}
\managertitle{高级工程师}
\managerarea{水处理工艺研发}
\managertelephone{18810012466}
\manageremail{my165888@163.com}
\attandent{张丰、史福有}
\expectperiod{2017.10-2018.5}
\expectfund{29万元}
\expectfundtype{公司自筹}
\achievementmiddle{中期:完成小试试验药剂购买,并搭建小试装置进行试验}
\achievementfinal{末期:给出小试试验报告}
\projectabstract{国家对工业园区监管力度逐渐加大,工业园区污水厂出水达标问题是一个亟待解决的棘手问题。采用常规的生化处理方法,很难将这类废水处理至排放标准。芬顿氧化技术是通过Fe$^{2+}$与H$_2$O$_2$之间发生反应产生·OH来降解水中的有机污染物,其具有氧化能力强、反应启动快、反应条件温和、反应设备简单、能耗消耗小、经济性好等优点,是针对工业园区理想的废水深度处理技术。

本项目采用芬顿氧化技术对工业园区二级生化出水进行深度处理,通过实验室小试试验优化工艺参数,减少污泥产量,降低运行成本,达到深度处理目标,确定一套适宜的工艺方案,并预备将该方案应用于不同工业园区废水的治理。
}
\projectobject{
    近年来,我国为了优化工业资源,建设了一批经济技术开发区、特色工业园区及技术示范区等多种形式的工业园区。2016年我国建成和在建的各类工业园区数量达到了9000多个,工业废水排放量占全国污水排放总量的45\%左右。工业园区污水主要来自园区工厂生产过程中产生的污水和废液,其主要特点是:成分复杂,污染物浓度高;具有一定毒性;可生化性较差;水质不稳定。基于成本考虑,我国对工业园区的污水大多仍采用“预处理+二级生化处理”的工艺,但随国家对出水水质标准的提高,单纯的生物处理工艺已很难满足出水水质要求,因此对工业废水二级生化出水进行深度处理,进一步降低有毒有害物质的含量具有重大的现实意义。近年来各种高级氧化技术,如紫外光催化氧化技术、臭氧催化氧化技术、电催化氧化技术、超声降解技术、芬顿催化氧化技术等在有毒难降解有机废水处理方面有大量的研究工作报道。

    芬顿氧化技术是通过Fe$^{2+}$与H$_2$O$_2$之间发生反应产生·OH来降解水中的有机污染物。·OH是强氧化性物质,能无选择地与废水中的污染物反应,降解常规生物方法无法处理的有毒有害物质,其在化工[1]、印染[2]、木薯制酒精[3]、含酚[4]、制药[5]等废水处理中均有应用,并取得了较好的处理效果。芬顿氧化技术处理工业废水具有反应启动快,反应条件温和、反应设备简单、能耗消耗小,经济性好等优点。虽然芬顿氧化技术具有上述优点,但其存在药剂消耗量大、污泥产量多的问题,这限制了其在工程上的大规模应用。
    本课题拟采用强化芬顿氧化技术深度处理工业园区二级生化出水,重点解决生化处理过程无法降解的有机物的处理问题,在保证出水水质合格的前提下,对芬顿氧化技术的工艺参数进行优化,以降低污泥产量与运行成本,为工业园区废水深度处理问题提供可行的解决方案,从而为芬顿氧化技术的工程化应用提供途径。
}
