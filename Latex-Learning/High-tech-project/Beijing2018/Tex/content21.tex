\par 煤的气化技术主要是指煤在高温环境之下,发生一定的化学反应,其内部化学性质发生变化,最终形成一氧化碳及氢气等物质,煤的气化技术在煤化工产业发展过程中是非常关键并且重要的。大部分的煤化工产业发展都离不开煤的气化。煤的气化技术分类根据不同的原理以及因素划分为不同的类型。\cite{huanghuijun_2016}根据燃料的具体运动状态和形式可以将煤的气化分为流化床煤气化方式、气流床煤的气化方式;根据进料方式则可以分为水煤浆气化方式,粉煤气化方式;根据温度的不同分为高温气化和低温气化;根据气化压力分为高压气化、常压气化以及低压气化等。总之,不同形式的气化有不同的分类标准。\par

(1)煤化工废水的来源以及特点简述\par
一般来说,煤化工废水分为多种不同的形式,按照常规分类方式可以将煤化工废水分为焦化废水、气化废水以及液化废水。焦化废水一般在传统煤化工生产过程中会产生,焦化废水一般都是由废物混合物以及水构成,在煤的焦化过程中,由杂质提炼之后所产生的废水等组成。煤的焦化废水其中含有的杂质和废弃物质含量和种类非常复杂,一般来说含有的氨氮元素较高,对环境会产生巨大的影响和危害。\cite{luming_2016}除此之外,也含有大量的含苯等多环有机物,这些有机物内部性质极其稳定,长时间留在土壤中也不会被微生物分解,在一定程度上大大增加了对环境的危害,而且这些多环有机物在一定程度上被人体吸收之后会致癌。焦化废水在现代化的煤化工生产加工过程中所产生的量较低,相比于传统煤化工生产来说,其对环境所产生的影响不高,这也是为什么当前大多数的传统煤化工产业逐渐向现代化煤化工产业转型的原因。\par
其次,气化废水主要构成物质为清洗煤炭过程中所产生的废水、预处理之后留下的杂质废水以及最终的蒸馏废水等。气化废水相比于焦化废水来说,虽然对环境产生的影响没有焦化废水高,但是其内部也含有大量的有害化合物,如油类物质。煤的气化废水如果没有经过深度处理之后排放到环境当中会严重影响到微生物的分解作用,其内部所含有的有机物会限制微生物的分解效果,进而降低土壤的利用效率。最后为煤的液化所产生废弃物质称之为液化废水。液化废水中也含有大量的污染物质,并且其中所含有的化学物质化学稳定性较好,传统的生化处理方式对其起不到有效作用。总而言之,煤化工产业发展和生产过程中会产生大量的污染物质,这些污染物质在一定程度上严重影响了我国经济的可持续发展,并且对生态环境会产生巨大的危害。煤的焦化废水、气化废水以及液化废水这三部分物质如果没有经过深度处理而直接排放到环境当中,会对水源会产生严重污染,这部分被严重污染的水源一旦进入到牲畜或者人的体内,将会造成不可估量的损失,甚至危害人的生命。而且当前来看,我国水源并不是非常丰富,如果对煤化工产业发展不加以有效的调节和控制,探索有效的废水处理技术,那么将对我国生态环境产生极大的损害。\cite{rongjunfeng_2015}\par

(2)国内外煤化工废水的处理现状\par
煤化工生产过程中所产生的工业废水的杂质含量非常高,并且所含有的杂质成分化学性质相对较稳定,在处理和清除的过程中需要多阶段多步骤共同进行才能够有效清除。当前国内外在煤化工生产废水的处理上有物理化学工艺,微生物处理、深度处理工艺等多种方式\par
①物理化学工艺\par
物理化学工艺主要是指借助物理方式和化学方式的共同作用来去除工业废水中的杂质。当前在煤化工废水处理中混凝法有广泛的应用。对于煤化工的焦化废水、气化废水以及液化废水的混凝剂的选择有氣化铝、聚合硫酸铁等物质,除了加入相应的混凝剂之外,还应当加入辅助物质来提髙混凝沉降的效率。具体的实验步骤为首先有关工作人员提前测量煤化工生产过程中产生的废水水量,大体确定废水中所含有杂质的含量以及种类等,然后确定具体的混凝剂,确定混凝剂加入的含量,废水中加入一定的混凝剂反应之后,运用物理方式加快反应的效率,物理方式可以通过搅拌、离心等方式达到。在废水中加入大量的混凝剂并且进行物理方式加快反应速度之后,大量溶于水的物质会与混凝剂形成较大的絮状物,这些颗粒较大的絮状物会在在水中分离,以不溶于水的物质出现,这些杂质通过过滤的方式就能够被去除。\cite{wangchunxu_2016}混凝法的核心原理就是依靠混凝剂的凝絮作用,结合物理方式达到充分反应并去除的效果。混凝法在当前煤化工废水处理中有着比较广泛的应用,但是煤化工废水不能单纯依靠混凝法来进行杂质清除,很容易遗漏废水中不与混凝剂反应的物质。第二种物理化学工艺为吹脱法,吹脱法主要原理为借助空气的作用来达到清除杂质的效果。当废水中相关杂质的含量较高时,超过气液平衡状态,那么可以通过向废水中通入空气,使液体中的杂质借助空气的作用进行清除。在实际的煤化工工业废水处理中,采用混凝法要显著高于吹脱法,其主要的原因为混凝法消除杂质的效率较高,并且杂质清除更加彻底。\par
②微生物处理方式\par
微生物处理方式在进行煤化工废水处理当中有着广泛的应用,并且有着良好的应用前景,微生物处理分为厌氧型处理方式及好氧型处理方式。大部分的微生物在进行废水杂质的处理过程中需要在无氧环境下进行,这也是为什么当前厌氧型水解酸化在废水处理中有着广泛的应用前景。微生物处理原理主要是借助微生物的作用,将废水中难以分解的杂质经过酶的作用分解成无害的物质,部分微生物能够将工业废水中的杂质分解为无害的物质;部分微生物则会将工业废水中杂质分解为易降解物质,后续再进行二次工艺深度处理。\cite{yaoshuo_2016}在当前来看,煤化工的焦化废水中大量的难以降解物质通过厌氧水解酸化能够有效去除,而且通过厌氧水解酸化之后,杂质被去除的基础之上,溶液的酸碱性达到平衡,废水中的其他物质为厌氧水解酸化菌种提供了一定的营养,提高了废水处理的效率。\cite{yuhai_2014}\par
③深度处理工艺\par
第三种处理工艺为煤化工废水的深度处理方式,很多情况下大量的煤化工工业废水如焦化废水、气化废水等都无法经过简单的处理之后就达到排放标准,而需要进行深度处理,将其中大量的难以降解的物质以及对环境产生损害的物质进行进一步处理。深度处理方式包括氧化还原处理、吸附处理、离子交换处理等。氧化处理方式主要借助氧化剂氧化作用或还原剂的还原作用,对废水中一些难降解物质进行氧化或还原,降低煤化工工业废水的污染性。\cite{zhangdong_2016}需要注意的是在进行氧化剂或还原剂的选择过程中需要防止二次污染现象出现,因为氧化还原反应会将氧化剂或还原剂中的离子进行置换,如果氧化剂或还原剂中离子含有一定的污染性,那么被置换之后进入到废水当中依然会对环境产生极大的损害,得不偿失。根据当前调查研宄显示,采用臭氧作为煤化工工业废水的氧化剂能够起到非常好的效果,不仅能够将废水中的杂质清除,同时也降低了引入二次污染的概率。其次,活性炭吸附方式,活性炭具有良好的吸附作用,不仅能够将废水中的杂质有效的吸附,同时也会吸收废水中难以处理的气味,杂质去除的效率非常高。但是目前国内在煤化工废水中的处理技术应用并不多见,主要是由于当前活性炭的价格和成本较高,不利于企业的经济效益。另一方面大部分的企业忽视煤化工废水的深度处理作用,直接将工业废水排放到自然界当中,影响了生态环境。W最后一种方式为膜分离技术,膜分离技术的分离原理为通过在废水中加入半透明的分离膜,将废水中大颗粒杂质进行处理,未经过处理的煤化工工业废水在经过半透膜之后,直径比半透膜通透性大的杂质会留在半透膜上,然而离子直径或分子直径小于半透膜通透性的物质则会穿过半透膜,这些穿过半透膜的物质最终形成的溶液在一定程度上对环境产生的影响要远远低于未经处理的废水溶液。而且通过半透膜分离技术,煤化工焦化废水、气化废水和液化废水产生的杂质也能够很好的利用起来,为企业的废物利用奠定良好的基础,提升企业的经济效益。\cite{zengsiyuan_2016}\par

④我国煤化工废水回用技术探究\par
回用技术对于煤化工废水不仅能够有效的进行处理,而且还能够废物再利用,大大提高了企业的可持续发展,促进了资源的回收利用效果。经过优化工厂的回用技术之后,废水当中的有害物质被提取出来,能够应用于其他工业生产。\par

a. 石灰软化水技术\par
煤化工废水中硬度主要是根据碳酸盐离子的含量决定,溶液中的Ca$^{2+}$和Mg$^{2+}$会形成碳酸盐,进而提髙溶液的硬度。通过石灰软化技术能够有效的降低溶液废水的硬度,提升溶液的软性,进而为后续的深度处理奠定良好的基础。将石灰乳加入到煤化工废水当中,其会与溶液中的Ca$^{2+}$和Mg$^{2+}$形成不溶于溶液的杂质,进而可以通过物理方式去除。\cite{zhangzhiwei_2013}\par

b. 反渗透技术\par
反渗透技术能够有效的将煤化工废水当中所含有的大量微生物、有害有机物以及难以降解的物质区分开来。主要利用的结构为低压复合膜,低压复合膜主要是通过多层膜片相叠加的方式形成,当工业废水流经低压复合膜时,废水会在膜片表面横向运动,然后既可以人工施加一定的压力,也可以借助废水自身的压力作用,一段时间之后,能够穿过低压复合膜的物质会经过膜进入到下侧,而分子颗粒较大的杂质和有害物质则停留在膜表面。最终的结果为流出低压复合膜为淡水,而留在低压复合膜以上的为浓度盐水,这些浓盐水经过工业的回收和处理之后应用在其他物质的生产上,实现了煤化工废水的回用效果。\cite{zhuanghaifeng_2016}\par

