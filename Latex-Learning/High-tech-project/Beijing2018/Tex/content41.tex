本项目为企业自主独立研究项目,用以考察催化氧化芬顿对\showproject的贡献,优化工艺运行参数,作为工业化设计的参考依据。研究重点放在H$_2$O$_2$、Fe$^{2+}$和催化剂投加量,反应水力停留时间及pH值对处理效果的影响之上。项目流程依据公司标准试验流程设计,作为研究目标的工业污水从\showprojectlocation处取得。项目分以下步骤进行,分别是技术调研、试验构建、最佳工艺控制参数确定、工程技术设计说明编制、生产性试验设计、试验验证。在管理上,项目严格按照公司指定的《科研课题管理办法》执行,将依次进行立项、定期小结、中期考核、验收评审的工作。