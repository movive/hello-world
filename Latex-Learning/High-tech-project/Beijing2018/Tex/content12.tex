芬顿氧化技术是通过Fe$^{2+}$与H$_2$O$_2$之间发生反应产生·OH来降解水中的有机污染物。此高级氧化反 应应用较为广泛,芬顿氧化法具有吸附性好,反应简单快速、可与杂质物质絮凝等特点,对芳烃类、酚类、芳胺类等难降解的有机废水效果较好。H$_2$O$_2$/Fe$^{2+}$体系氧化主要依靠链反应催化生成的羟基自由基,在当前污水处理中,已知应用最强的氧化剂是 OH·。芬顿试剂反应产生的 OH·自由基具有重要性质,OH·自由基反应氧化性强,OH·自由基反应选择性小,H$_2$O$_2$分解成 OH·自由基的反应极为迅速,可以氧化多数有机物。与传统污水处理方法相比,OH·自由基能够氧化绝大多数有机物,并且可以使整个链反应顺利进行,因此芬顿氧化法对去除传统煤化工废水中难以去除的难降解有机物具有明显的优势。此外芬顿试剂也是十分常见的试剂,因此更易操作从而良好的经济效益。芬顿氧化法具有独特优势,OH·自由基同时还可以发生加成反应。反应机理复杂,羟基自由基与有机物反应生成游离基,并进一步氧化生成 CO$_2$和 H2O,煤化工污水中的 COD 含量可以得到有效的降低。芬顿试剂一般在酸性条件下使用,所以 Fe(OH)$_3$以胶体形态存在,故具有较好的吸附与凝聚的能力,因而对去除水中部分悬浮物及杂质有良好的效果。结合国内外对煤化工废水深度处理的方法以及我国煤化工废水生化出水的特点,经过比较混凝沉淀法、芬顿氧化法等对煤化工废水的处理效果,芬顿氧化对于煤化工污水中 COD有更为显著的作用和良好的去除效果。芬顿试剂在深度处理煤化工废水物的实际工程应用中有待于进一步的详细深入的研究反应机理,芬顿氧化对废水的处理仍需进一步实验。\par
本课题拟探索强化芬顿氧化技术技术在\showproject中的应用,在保证出水水质合格的前提下,对芬顿氧化技术的工艺参数进行优化,摸索出成套化设备的设计方法,为\showproject提供可行的解决方案,从而为芬顿氧化技术的工程化应用提供途径。\par

