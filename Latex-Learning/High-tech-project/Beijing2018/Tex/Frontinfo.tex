%---------------------------------------------------------------------------%
%->> Titlepage information
%---------------------------------------------------------------------------%
%-
%-> 中文封面信息
%-

%---------------------------------------------------------------------------%
%---总体信息
\projectname{煤炭行业生产端废水无害化处理方式研究}
\id{RD03}
\project{煤炭行业生产端废水无害化处理技术}
\projectlocation{安徽铜陵城北污水处理厂}
\manager{解建坤}
\department{设计研究院}
\company{北京桑德环境工程有限公司}
\managergender{男}
\managerbirth{1982.03}
\managerethnic{汉}
\managerdegree{博士}
\managertitle{高级工程师}
\managerarea{水处理工艺研发}
\managertelephone{15810355385}
\manageremail{xiejiankun@soundglobal.cn}

%---开题信息
\attendant{许慧英、仝延忠、白利云、彭永立、张锦祥}
\expectperiod{2018年1月-2018年12月}
\fund{900万元}
\fundtype{公司自筹}
\achievementmiddle{中期:初步取得处理煤炭行业生产端废水的Fenton试剂投加量}
\achievementfinal{末期:在生产中验证工艺处理煤炭行业生产端废水的稳定性}
\projectarea{水处理工艺开发}
\projectdirection{工业废水等难降解污水深度处理}
\projectabstract{\setlength{\parindent}{2em}\qquad煤炭行业生产端废水多种多样,有煤矿开采中产生的煤层气废水,如压裂液和采气排水,也有煤炭制焦过程产生的焦化废水,这系列的废水具有高矿化度,高盐度,难以生物处理等特点,若不经处理直接排放,必然对环境产生恶劣影响。本次主要针对煤炭行业生产端废水进行工艺开发,确定处理该类废水的主要工艺方案,并通过设计、制造、安装、调试运行等工作,在煤炭行业生产端废水处理领域建立一套完整的水处理体系,保证运行出水结果稳定达标,为公司在煤化工行业打开新的市场。
}

%---结题信息
\attendantchanged{许慧英、仝延忠、白利云、彭永立、张锦祥(无变化)}
\actualperiod{2018年1月-2018年12月}
\actualfund{860.97万元}
\actualfundtype{公司自筹}
\actualachievementmiddle{中期:初步取得药剂投加量,并对实验进行优化.}
\actualachievementfinal{末期:结合生产,验证Fenton工艺处理焦化废水的稳定性}
\jietiabstract{\par强化芬顿技术为核心的高级催化氧化工艺适用于煤化工废水处理领域,可解决\showproject中存在的难题。课题中发现焦化废水水质水量波动大,以水质形状作为投药和调节pH的指标既满足处理效果又降低运行成本。此外,水质中TDS极高并含有高浓度的硫化物易产生浮泥和发臭问题,需降低沉淀池表面负荷,延长沉淀池的停留时间,同时投加硫酸亚铁,可去除水中的硫化物及少部分的还原性物质。\par
}
