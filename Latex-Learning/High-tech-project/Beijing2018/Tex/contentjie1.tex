课题经过小规模试验和生产性验证,得到以下研究成果:\par
1)焦化废水水质水量波动大,尤其以黑色水质的影响最大,该类废水可作为加药量调节指标。\par
2)反应pH值对Fenton反应的影响较大,白色透明水质,酸度越低,处理效果越差,可根据来水形状调整反应pH值达到最优效果的同时降低处理成本。\par
3)水质中TDS极高,易产生浮泥,需降低沉淀池表面负荷,延长沉淀池的停留时间;\par
4)废水中含高浓度的硫化物,使其产生发臭问题。不加酸碱的条件下,直接投加硫酸亚铁,可去除水中的硫化物及少部分的还原性物质。\par

