\par
\qquad近几年来,由于工业生产的快速扩张,诸如酸雨、雾霾等恶劣的环境问题相继爆发,我国的生态环境面临着前所未有的打击,而在此之中,煤化工产业对环境造成的破坏更是尤为严重。近几年来,煤化工产业发展迅猛,对生态环境造成了巨大的破坏,其中对水资源造成的破坏更是重中之重,这就要求我们不得不重视煤化工生产的污水排放问题。同时由于国内的煤化工产地主要集中在西北地区,这一地区水资源相对匮乏,而由于煤化工造成的水资源污染,更是加剧了该地区的水资源短缺问题,因而对于现代煤化工生产工程中产生的污水进行合理的回收处理,实现废水的零排放对于缓解水资源短缺、保护环境、推进现代煤化工的可持续发展都是非常重要的。\par煤化工行业主要由传统煤化工行业以及现代化煤化工行业两种形式构成。现代化煤化工行业主要是指以煤气化等为主要生产原理的企业,通过煤的气化来生产天然气等污染性较小的产物。传统煤化工则主要是通过煤的焦化来进行生产,总结来说,传统煤化工产业的发展在一定程度上产生的废水以及污染物要远远的高于现代化煤化工产业。而且当前大部分的煤化工产业发展都逐渐由传统煤化工行业转为现代化的煤化工行业,因为现代化煤化工行业不仅在生产效率上要高于传统煤化工产业生产,同时在废水的排放和回收再利用上也要远远高于传统煤化工产业。现阶段中国工业的发展决定了这两种煤化工行业在接下来的十年内共生共存。这给\showproject带来一定挑战,处理工艺需要考虑现代化煤化工污水水质特征,处理出水满足工业回用标准,也要兼容传统煤化工水处理工艺,以经济有效的改造工程助其追上日益严格的排放要求。
